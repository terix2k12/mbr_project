%%%%%
%
% Indian Institute of Technology Madras, Chennai
%
% CS Memory Based Reasoning in AI
%   
% autor:   Philipp Fonteyn
% version: 0.1 - 6 March 2012
% created: 06. March 2012
%
%%%%%
\documentclass[10pt,oneside,a4paper]{scrartcl}

%
% Preamble and header 
%
\newcommand{\courseTitle}{Memory Based Reasoning in AI}
\newcommand{\creationDate}{6th of March 2012}
\newcommand{\semester}{January - May 2012}
\newcommand{\profName}{S CHakraborti}
\input{../../uni-preamble}
\begin{document}	
\input{../../uni-preface}

%
% Document
%

\section{Reasoning and Rationality}
MBR in AI
- reasoning
- computation study of humasn vs machine memorz
- human memory
- reasoning influences memory on VSV

includes fields of 
cognitive science
neuro science
psychology
statistics
AI


Experience Resuse
A lot of advantages follow
	shorter time, imrpoved quality, less knowledge

goedelstheorem incompletness theorem -> no perfect world knowledge representation


Is it always a good idea to imitate humans? (natures)  ----engineeering---> wheel

abstractions 
     traditional AI
     "soft ai" not based solemnly on logic

memories are creating expectatinos
-rules of a game derive reasoning
- memories add/modify new rules


"ways of thinking" by lazlo mero


what is (not) acting rationaly
-having faith
-not knowing procedure
- bias
-looks promosing solution               what is "rational"

memory -> learning  -> language -> memory

types of knowledge (roger schank)
- rational
- non conscious
-physically conscious
-emotionally conscious

"The 4 views of AI"
act \ think   human   rationally

human         x                     x

rationally     x                   v


Evolution of language (laguage mamory interaction)
 we have to learn their language

existing language

memory
language
interaction

combines to structure funtion correspondence

-reductionist view
-systemic view ---> grow systems not build them







Edit distances on graphs
transformation approaches                might improve similar problems closer to similar solutions


a complexity measure??

retrieval vs adaption tradeoff


1)) weight correlation
2)) cohesion           Luc lauritsen


alignment measure MASSIVE

picture2

similarity measure for c_n
yields closest in P is closest in S

applications choose right representation
case base mainenance (complexity guided) competence modelling






Expectation failure
- you do not get what you expected, thus your model is incorrect
- update with certain observations
- when you get more than one pointer to a certain change, do it





Generalized Cases
- point case d,l
- constant case d,l \in D
- function generalized d,f(d).



\section{CBR}
case based reasoning


\section{the idea of compression}
spring-algorithm
cluster inbetween distance

compressing you more you know you more you compress

compressing you more you know you more you compress

pnemonics
chunking
\section{Retrieval}

complexity of retrieval
internal complexity
external complexity   there might be a tradeoff
Goals
1) avoid multile local sim computations with same a.r.
2) avoid distance computations with case that have 0 similarity


\subsection{K-d trees }
A kd-tree is a k-dimensional binary search tree. All cases are leaves in this tree and the other nodes partition the case base following certain decision aspects.

Important is, that you can order the cases regarding the attribute you are making a decision on.


\cite{wess 95 althoff derwand}


\subsection{Inreca Trees}


\subsection{Fish and Shrink}


cases are seperated into complex attributes called aspects
flexile weight for each aspect, defined by query
overall similaritz measure is defined as sum over weighted aspects

Is well suited for complex similarities but small case bases
measures need to fullfill triangle inequaility

\cite{schaaf}



\subsection{Case Retrieval Nets / FCRNs}
burkahrd and lenz

Idee
Aufbau eines Netzes (vgl. Neuronales Netz) zwecks Zugriff
o Zerlegen der Fallinformationen in Informationseinheiten (IEs)
(z.B. Attribut-Wert-Paar)
- Jede Informationseinheit wird ein Netzknoten
- Jeder Fall wird durch einen Fallknoten repräsentiert
o Informationseinheiten, zwischen denen eine Ähnlichkeit >0 besteht,
werden verbunden.
- Verbindungsstärke = Ähnlichkeit.
o Zum Retrieval werden die Informationseinheiten der Anfrage aktiviert.
o Die Aktivität wird durch das Netzwerk bis zu den Fallknoten
propagiert.
o Aktivität an den Fallknoten spiegelt die Ähnlichkeit zur Anfrage wider.

Grundlegend für das CRN ist: Eine Informationseinheit
(IE) ist eine atomare Wissenseinheit
- z.B. Attribut-Wert-Paar
- Für numerische Werte mehrere IEs
- Text: Pro Stammform ein IE
o Ein Fall wird durch eine Menge von Informationseinheiten
repräsentiert
- z.B. AVPs, oo-Repräsentationen, Texte
o Eine Query (Anfrage) besteht auch aus einer Menge von
Informationseinheiten




A CRN is defined as a 5-tupel $N=(E,C,\sigma,\rho,\Pi)$
\item{E is a finite set of IE}
\item{C is a finite set of case nodes}
\item{$\sigma$ is a similarity function}
\item{$\rho$ is a relevance function}
\item{$\Pi$ is a set of progapation functions $\pi_n:IR^E->IR \forall n \in E \cup C$}





\subsection{Inverted File}
should be as measure?








index based retrieval
frist generates an index structure










basic case retreival network BCRN
mario lenz "information entity"




\section{case base basics}

Ralph Bergmann - Experience Managemetn

CBR
D =
L =
case i = di li
description lesson


utility u:DxL   how useful is lk to solve dk



distinguishability
p is a problem
if di=dk then u(di,lm)=u(dk,lm)



similarity measures DxD auf 0,1  sim(di,dj)
similar problems have similar solutinos
sim(d,ki) > sim(d,kj( -> u dki > u d kj



retrieval mechanisms
representation/indexing
learning/evolution


evolutionery needs }
-strucutre functions
-language evolution

weak } reductionist -> system science [whole > sum of parts]


\subsection{similarity measures}



                                                    

notesv1.png


notesv2.png




%
% Literature
%


\bibliographystyle{alpha}
\bibliography{thebib}

%
% End of File
%
\end{document}
